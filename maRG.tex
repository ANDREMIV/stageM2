\documentclass[10pt, a4paper]{report}
\usepackage{amsmath}
\numberwithin{equation}{subsection}
\usepackage{graphicx}
\usepackage{titlesec}
\usepackage{appendix}
\usepackage[french]{babel}
\usepackage{xcolor,graphicx}
\usepackage[top=0.6in,bottom=0.6in,right=1in,left=1in]{geometry}
\usepackage{cite}
\usepackage{tocbibind}
\usepackage{siunitx}
\sisetup{output-exponent-marker=\ensuremath{\mathrm{e}}}
\usepackage{verbatim}
\usepackage{subcaption}
\usepackage{listings}
\begin{comment}
\usepackage{etoolbox}
\makeatletter
\patchcmd{\chapter}{\if@openright\cleardoublepage\else\clearpage\fi}{}{}{}
\makeatother
\end{comment}

\usepackage{ifpdf}
\ifpdf
\usepackage[pdftex]{hyperref}
\else
\usepackage{hyperref}
\fi
\renewcommand{\thesection}{\arabic{section}}

\begin{document}

\begin{titlepage}

\begin{center}


% Title
\rule{\linewidth}{0.3mm} \\[0.4cm]
{ \huge \bfseries\color{blue!70!black} Cours d'introduction à la Relativité Générale\\[0.4cm] }
\rule{\linewidth}{0.3mm} \\[0.4cm]
\Large Miville André
\normalsize

\vfill

% Bottom of the page

\end{center}
\end{titlepage}







%\maketitle
\renewcommand{\contentsname}{Sommaire}
\renewcommand{\bibname}{Références}



%\addcontentsline{toc}{chapter}{Quelques rappels de Relativité Générale} 
\tableofcontents


\section{Généralités}
La relativité générale est une théorie décrivant la forme de l'espace temps par la théorie mathématique des variétés courbes. Le principe d'équivalence d'Einstein est la base physique de la théorie. Il dit que l'observateur en chute libre, c'est à dire celui accéléré par le champ gravifique est localement inertiel, i.e. la relativité restreinte s'applique. Lorsqu'on tente d'appliquer ce principe, on s'aperçoit que cela n'est pas possible dans un espace-temps plat mais possible dans une autre théorie comme celle des variétés courbes. Dans ce formalisme, on munie une variété différentielle par une connexion affine qui connecte les espaces tangents. En relativité générale on utilise la connexion de Lévi-Civita. La "distance" utilisée est l'intervalle d'espace temps qui mesure physiquement le temps propre pour l'observateur comobile à la ligne d'univers considérée. 

\section{Coordonnées généralisés}
\subsection{base naturelle}


\section{Connexion}
\subsection{Généralités}
Sur toute variété, on peut définir une infinité de connexions affines. Le choix d'une connexion affine est équivalent à définir une façon de dériver les champs de vecteurs. Un choix de connexion affine est aussi équivalent à une notion de transport parallèle, c'est-à-dire à un moyen de transporter les vecteurs le long de courbes de la variété. Les principaux invariants d'une connexion affine sont sa courbure et sa torsion.


\subsection{Coefficients $\&$ symboles de Christoffels}
La coefficients d'une connexion s'expriment dans une base arbitraire par :
\begin{equation}
\boxed{\nabla_{\mathbf{u}_i}\mathbf{u}_j = {\omega^k}_{ij}\mathbf{u}_k}
\end{equation}
Explicitement pour la connexion de Lévi-Civita en terme du tenseur métrique :

\begin{equation}
\boxed{{\omega^i}_{kl} = \frac{1}{2} g^{im} \left( g_{mk,l} + g_{ml,k} - g_{kl,m} + c_{mkl} + c_{ml k} - c_{kl m} \right)}
\end{equation}
Où $[\mathbf{u}_k,\, \mathbf{u}_l] = {c_{kl}}^m \mathbf{u}_m$ avec $\mathbf{u}$ les vecteurs de base et $[]$ le crochet de Lie.
Si l'on choisit la base naturelle des coordonnées généralisées, l'expression des coefficients de la connexion de Lévi-Civita se simplifie et on les appelle symboles de Christoffels. L'utilisation de symboles de Christoffels par la suite rend implicite l'usage de la connexion de Lévi-Civita et dans la base naturelle.
\begin{equation}
\boxed{{\Gamma^i}_{kl} = \frac{1}{2} g^{im} \left(g_{mk,l} + g_{ml,k} - g_{kl,m}\right)}
\end{equation}
 Sous un changement de coordonnées généralisées passant de $\left(x^1,\, \ldots,\, x^n\right)$ à $\left(\bar{x}^1,\, \ldots,\, \bar{x}^n\right)$, les symboles de Christoffels se transforment ainsi :
\begin{equation}
\boxed{{\bar{\Gamma}^i}_{kl} =
  \frac{\partial \bar{x}^i}{\partial x^m}\,
  \frac{\partial x^n}{\partial \bar{x}^k}\,
  \frac{\partial x^p}{\partial \bar{x}^l}\,
  {\Gamma^m}_{np}
  + 
  \frac{\partial^2  x^m}{\partial \bar{x}^k \partial \bar{x}^l}\,
  \frac{\partial \bar{x}^i}{\partial x^m}}
\end{equation}

\section{Géodésiques}
\subsection{Géodésique de la connexion}


\subsection{Géodésique de la métrique}
La géodésique de la métrique est la courbe qui extrémalise la distance entre deux points de la variété. La géodésique d'une sphère est un grand cercle.

L'équation des géodésiques de la métrique prend une forme simplifiée en paramétrisant la courbe par la distance ce qui revient en RG à paramétriser la ligne d'univers par le temps propre :

\begin{equation}
\boxed{\frac{1}{2} \left(g_{ij,k} - g_{ki,j} - g_{kj,i}\right)\dot{x}^i \dot{x}^j- g_{ki}\ddot{x}^i= 0}
\end{equation}

Ce qui se réécrit avec les symboles de Christoffels :
\begin{equation}
\boxed{\ddot{x}^k + \Gamma^k_{ij} \dot{x}^i\dot{x}^j=0}
\end{equation}



\section{Tenseur de Courbure}
Le Tenseur de Courbure ou de Riemann vaut : 
\begin{equation}
\boxed{ R(u,v)w =\nabla_u\nabla_v w - \nabla_v \nabla_u w -\nabla_{[u,v]} w }
\end{equation}
Il exprime la non commutativité de la dérivée covariante. Lorsqu'on transporte parallèlement un champ de vecteur w autour d'une boucle défini par les champs de vecteur u et v, on obtient un changement des composantes w en un point de la variété qui s'exprime grâce à ce tenseur.

En RG on obtient :
\begin{equation}
\boxed{{R^\sigma}_{\mu\nu\kappa} =
  {\partial{\Gamma^\sigma}_{\mu\kappa} \over \partial x^\nu} -
  {\partial{\Gamma^\sigma}_{\mu\nu} \over \partial x^\kappa} +
  {\Gamma^\sigma}_{\nu\lambda}{\Gamma^\lambda}_{\mu\kappa} -
  {\Gamma^\sigma}_{\kappa\lambda}{\Gamma^\lambda}_{\mu\nu}}
\end{equation}

La contraction du tenseur de Riemann nous donne le tenseur de Ricci : 
\begin{equation}
\boxed{R_{ij} = {R^k}_{ikj}}
\end{equation}

La contraction du tenseur de Ricci nous donne la courbure scalaire : 
\begin{equation}
\boxed{R  = g^{ij}R_{ij} = R^j_j}
\end{equation}

\section{Distance physique}
Pour évaluer une distance physique, on peut un r


\end{document}

\end{document}